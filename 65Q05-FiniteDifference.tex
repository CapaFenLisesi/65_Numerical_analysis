\documentclass[12pt]{article}
\usepackage{pmmeta}
\pmcanonicalname{FiniteDifference}
\pmcreated{2013-03-22 15:35:00}
\pmmodified{2013-03-22 15:35:00}
\pmowner{CWoo}{3771}
\pmmodifier{CWoo}{3771}
\pmtitle{finite difference}
\pmrecord{11}{37493}
\pmprivacy{1}
\pmauthor{CWoo}{3771}
\pmtype{Definition}
\pmcomment{trigger rebuild}
\pmclassification{msc}{65Q05}
\pmrelated{Equation}
\pmrelated{RecurrenceRelation}
\pmrelated{IndefiniteSum}
\pmrelated{DifferentialPropositionalCalculus}
\pmdefines{forward difference}
\pmdefines{backward difference}
\pmdefines{difference equation}

\usepackage{amssymb,amscd}
\usepackage{amsmath}
\usepackage{amsfonts}

% used for TeXing text within eps files
%\usepackage{psfrag}
% need this for including graphics (\includegraphics)
%\usepackage{graphicx}
% for neatly defining theorems and propositions
%\usepackage{amsthm}
% making logically defined graphics
%%%\usepackage{xypic}

% define commands here
\begin{document}
\textbf{Definition of $\Delta$.}

The derivative of a function $f\colon\mathbb{R}\to\mathbb{R}$ is defined to be the expression
$$\frac{df}{dx}:=\lim_{h\to 0}\frac{f(x+h)-f(x)}{h},$$
which makes sense whenever $f$ is differentiable (at least at $x$).
However, the expression
$$\frac{f(x+h)-f(x)}{h}$$
makes sense even without $f$ being continuous, as long as $h\neq 0$.
The expression is called a \emph{finite difference}.  The simplest
case when $h=1$, written
$$\Delta f(x):=f(x+1)-f(x),$$
is called the \emph{forward difference} of $f$.  For other non-zero
$h$, we write
$$\Delta_h f(x):=\frac{f(x+h)-f(x)}{h}.$$  When $h=-1$, it is called
a \emph{backward difference} of $f$, sometimes written $\nabla
f(x):=\Delta_{-1} f(x)$.  
Given a function $f(x)$ and a real number $h\neq 0$, if we define $y=\frac{x}{h}$ and $g(y)=\frac{f(hy)}{h}$, then we have 
$$\Delta g(y)=\Delta_h f(x).$$  
Conversely, given $g(y)$ and $h\neq 0$, we can find $f(x)$ such that $\Delta g(y)=\Delta_h f(x)$.

\textbf{Some Properties of $\Delta$.}  

It is easy to see that the forward difference operator $\Delta$ is linear:
\begin{enumerate}
\item $\Delta(f+g)=\Delta(f)+\Delta(g)$
\item $\Delta(cf)=c\Delta(f)$, where $c\in\mathbb{R}$ is a
constant.
\end{enumerate}
$\Delta$ also has the properties
\begin{enumerate}
\item $\Delta(c)=0$ for any real-valued constant function $c$, and
\item $\Delta(I)=1$ for the identity function $I(x)=x$.
constant.
\end{enumerate}
The behavior of $\Delta$ in this respect is similar to that of the
derivative operator.  However, because the continuity of $f$ is not assumed, $\Delta f=0$ does not imply that $f$ is a constant.  $f$ is merely a periodic function $f(x+1)=f(x)$.
Other interesting properties include
\begin{enumerate}
\item $\Delta a^x=(a-1)a^x$ for any real number $a$
\item $\Delta x^{(n)}=nx^{(n-1)}$ where $x^{(n)}$ denotes the falling factorial polynomial
\item $\Delta b_n(x)=nx^{n-1}$, where $b_n(x)$ is the Bernoulli polynomial of order $n$.
\end{enumerate}

From $\Delta$, we can also form other operators.  For example, we
can iteratively define
\begin{eqnarray}
&&\Delta^{1}f:=\Delta f \\
&&\Delta^{k}f:=\Delta(\Delta^{k-1}f),\quad\mbox{where
}k>1.
\end{eqnarray}
Of course, all of the above can be readily generalized to $\Delta_h$.
It is possible to show that $\Delta_h f$ can be written as a linear combination of $$\Delta f,\Delta^2 f,\ldots,\Delta^h f.$$

\textbf{Difference Equation.}

Suppose $F\colon\mathbb{R}^n\to\mathbb{R}$ is a real-valued function
whose domain is the $n$-dimensional Euclidean space.  A
\emph{difference equation} (in one variable $x$) is the equation of
the form
$$F(x,\Delta_{h_1}^{k_1}f,\Delta_{h_2}^{k_2}f,\ldots,\Delta_{h_n}^{k_n}f)=0,$$
where $f:=f(x)$ is a one-dimensional real-valued function of $x$.
When $h_i$ are all integers, the expression on the left hand side of
the difference equation can be re-written and simplified as
$$G(x,f,\Delta f,\Delta^{2}f,\ldots,\Delta^{m}f)=0.$$
Difference equations are used in many problems in the real world,
one example being in the study of traffic flow.
%%%%%
%%%%%
\end{document}
