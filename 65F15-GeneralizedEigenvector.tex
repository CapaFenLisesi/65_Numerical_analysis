\documentclass[12pt]{article}
\usepackage{pmmeta}
\pmcanonicalname{GeneralizedEigenvector}
\pmcreated{2013-03-22 17:23:13}
\pmmodified{2013-03-22 17:23:13}
\pmowner{CWoo}{3771}
\pmmodifier{CWoo}{3771}
\pmtitle{generalized eigenvector}
\pmrecord{13}{39754}
\pmprivacy{1}
\pmauthor{CWoo}{3771}
\pmtype{Definition}
\pmcomment{trigger rebuild}
\pmclassification{msc}{65F15}
\pmclassification{msc}{65-00}
\pmclassification{msc}{15A18}
\pmclassification{msc}{15-00}
\pmrelated{GeneralizedEigenspace}
\pmdefines{cycle of generalized eigenvectors}

\endmetadata

\usepackage{amssymb,amscd}
\usepackage{amsmath}
\usepackage{amsfonts}
\usepackage{mathrsfs}

% used for TeXing text within eps files
%\usepackage{psfrag}
% need this for including graphics (\includegraphics)
%\usepackage{graphicx}
% for neatly defining theorems and propositions
\usepackage{amsthm}
% making logically defined graphics
%%\usepackage{xypic}
\usepackage{pst-plot}
\usepackage{psfrag}

% define commands here
\newtheorem{prop}{Proposition}
\newtheorem{thm}{Theorem}
\newtheorem{ex}{Example}
\newcommand{\real}{\mathbb{R}}
\newcommand{\pdiff}[2]{\frac{\partial #1}{\partial #2}}
\newcommand{\mpdiff}[3]{\frac{\partial^#1 #2}{\partial #3^#1}}
\begin{document}
Let $V$ be a vector space over a field $k$ and $T$ a linear transformation on $V$ (a linear operator).  A non-zero vector $v\in V$ is said to be a \emph{generalized eigenvector} of $T$ (corresponding to $\lambda$) if there is a $\lambda\in k$ and a positive integer $m$ such that $$(T-\lambda I)^m(v)=0,$$
where $I$ is the identity operator.

In the equation above, it is easy to see that $\lambda$ is an eigenvalue of $T$.  Suppose that $m$ is the least such integer satisfying the above equation.  If $m=1$, then $\lambda$ is an eigenvalue of $T$.  If $m>1$, let $w=(T-\lambda I)^{m-1}(v)$.  Then $w\ne 0$ (since $v\ne 0$) and $(T-\lambda I)(w)=0$, so $\lambda$ is again an eigenvalue of $T$.

Let $v$ be a generalized eigenvector of $T$ corresponding to the eigenvalue $\lambda$.  We can form a sequence $$v, (T-\lambda I)(v), (T-\lambda I)^2(v), \ldots, (T-\lambda I)^i(v), \ldots, (T-\lambda I)^m(v)=0, 0, \ldots$$
The set $C_{\lambda}(v)$ of all non-zero terms in the sequence is called a \emph{cycle of generalized eigenvectors} of $T$ corresponding to $\lambda$.  The cardinality $m$ of $C_{\lambda}(v)$ is its \PMlinkescapetext{\emph{length}}.  For any $C_{\lambda}(v)$, write $v_{\lambda}=(T-\lambda I)^{m-1}(v)$.

Below are some properties of $C_{\lambda}(v)$:
\begin{itemize}
\item 
$v_{\lambda}$ is the only eigenvector of $\lambda$ in $C_{\lambda}(v)$, for otherwise $v_{\lambda}=0$.
\item 
$C_{\lambda}(v)$ is linearly independent.
\begin{proof}  Let $v_i=(T-\lambda I)^{i-1}(v)$, where $i=1,\ldots,m$.  Let $0=\sum_{i=1}^m r_iv_i$ with $r_i\in k$.  Induct on $i$.  If $i=1$, then $v_1=v\ne 0$, so $r_1=0$ and $\lbrace v_1\rbrace$ is linearly independent.  Suppose the property is true when $i=m-1$.  Apply $T-\lambda I$ to the equation, and we have $0=\sum_{i=1}^m r_i(T-\lambda I)(v_i)= \sum_{i=1}^{m-1}r_iv_{i+1}$.  Then $r_1=\cdots=r_{m-1}=0$ by induction.  So $0=r_mv_m=r_mv_{\lambda}$ and thus $r_m=0$ since $v_{\lambda}$ is an eigenvector and is non-zero.
\end{proof}
\item
More generally, it can be shown that $C_{\lambda}(v_1)\cup \cdots \cup C_{\lambda}(v_k)$ is linearly independent whenever  $\lbrace v_{1\lambda},\ldots,v_{k\lambda}\rbrace$ is.
\item
Let $E=\operatorname{span}(C_{\lambda}(v))$.  Then $E$ is a $(m+1)$-dimensional subspace of the generalized eigenspace of $T$ corresponding to $\lambda$.  Furthermore, let $T|_E$ be the restriction of $T$ to $E$, then $[T|_E]_{C_{\lambda}(v)}$ is a Jordan block, when $C_{\lambda}(v)$ is ordered (as an ordered basis) by setting $$(T-\lambda I)^i(v)<(T-\lambda I)^j(v)\qquad\mbox{ whenever }\qquad i>j.$$  Indeed, for if we let $w_i=(T-\lambda I)^{m+1-i}(v)$ for $i=1,\ldots m+1$, then
\begin{eqnarray*}
T(w_i)=(T-\lambda I + \lambda I)(T-\lambda I)^{m+1-i}(v) &=& \left\{
\begin{array}{ll}
\lambda w_i & \mbox{ if } i=1, \\
w_{i-1}+\lambda w_i & \mbox{ otherwise.}
\end{array}\right.
\end{eqnarray*}
so that $[T|_E]_{C_{\lambda}(v)}$ is the $(m+1)\times (m+1)$ matrix given by
$$\begin{pmatrix}
\lambda & 1 & 0 & \cdots & 0\\
0 & \lambda & 1 & \cdots & 0\\
0 & 0 & \lambda & \cdots & 0\\
\vdots & \vdots & \vdots & \ddots & \vdots\\
0 & 0 & 0 & \cdots & 1\\
0 & 0 & 0 & \cdots & \lambda
\end{pmatrix}$$
\item
A cycle of generalized eigenvectors is called \emph{maximal} if $v\notin (T-\lambda I)(V)$.  If $V$ is finite dimensional, any cycle of generalized eigenvectors $C_{\lambda}(v)$ can always be extended to a maximal cycle of generalized eigenvectors $C_{\lambda}(w)$, meaning that $C_{\lambda}(v)\subseteq C_{\lambda}(w)$.  
\item
In particular, any eigenvector $v$ of $T$ can be extended to a maximal cycle of generalized eigenvectors.  Any two maximal cycles of generalized eigenvectors extending $v$ span the same subspace of $V$.
\end{itemize}

\begin{thebibliography}{3}
\bibitem{Friedberg} Friedberg, Insell, Spence. {\it Linear Algebra}. Prentice-Hall Inc., 1997.
\end{thebibliography}
%%%%%
%%%%%
\end{document}
