\documentclass[12pt]{article}
\usepackage{pmmeta}
\pmcanonicalname{LagrangeInterpolationFormula}
\pmcreated{2013-03-22 11:46:21}
\pmmodified{2013-03-22 11:46:21}
\pmowner{drini}{3}
\pmmodifier{drini}{3}
\pmtitle{Lagrange interpolation formula}
\pmrecord{16}{30229}
\pmprivacy{1}
\pmauthor{drini}{3}
\pmtype{Theorem}
\pmcomment{trigger rebuild}
\pmclassification{msc}{65D05}
\pmclassification{msc}{41A05}
\pmsynonym{Lagrange's Interpolation formula}{LagrangeInterpolationFormula}
%\pmkeywords{Interpolation}
%\pmkeywords{Extrapolation}
%\pmkeywords{Polynomial}
%\pmkeywords{Derivative}
\pmrelated{SimpsonsRule}
\pmrelated{LectureNotesOnPolynomialInterpolation}
\pmdefines{Lagrange polynomial}

\usepackage{amssymb}
\usepackage{amsmath}
\usepackage{amsfonts}
%\usepackage{graphicx}
%%%%%\usepackage{xypic}
\begin{document}
Let $(x_1,y_1), (x_2,y_2),\dotsc,(x_n,y_n)$ be $n$ points in the plane ($x_i\neq x_j$ for $i\neq j$). Then there exists a unique polynomial $p(x)$ of degree at most $n-1$ such that $y_i=p(x_i)$ for $i=1,\ldots,n$.

Such polynomial can be found using \emph{Lagrange's interpolation formula}:

\[
p(x)=\frac{f(x)}{(x-x_1)f'(x_1)}y_1+\frac{f(x)}{(x-x_2)f'(x_2)}y_2+\cdots+\frac{f(x)}{(x-x_n)f'(x_n)}y_n
\]
where $f(x)=(x-x_1)(x-x_2)\cdots(x-x_n)$.

To see this, notice that the above formula is the same as

\begin{align*}
 p(x) &=  y_1 \frac{(x-x_2)(x-x_3)\dots(x-x_n)}{(x_1-x_2)(x_1-x_3)\dots(x_1-x_n)} +
y_2 \frac{(x-x_1)(x-x_3)\dots(x-x_n)}{(x_2-x_1)(x_2-x_3)\dots(x_2-x_n)}\\
&\phantom{=}\qquad+\dots+y_n \frac{(x-x_1)(x-x_2)\dots(x-x_{n-1})}{(x_n-x_1)(x_n-x_2)\dots(x_n-x_{n-1})}
\end{align*}

and that for all $x_i$, every numerator except one vanishes, and this numerator will be identical to the denominator, making the overall quotient equal to 1.  Therefore, each $p(x_i)$ equals $y_i$.
%%%%%
%%%%%
%%%%%
%%%%%
\end{document}
