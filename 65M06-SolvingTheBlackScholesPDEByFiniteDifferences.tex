\documentclass[12pt]{article}
\usepackage{pmmeta}
\pmcanonicalname{SolvingTheBlackScholesPDEByFiniteDifferences}
\pmcreated{2013-03-22 16:30:59}
\pmmodified{2013-03-22 16:30:59}
\pmowner{stevecheng}{10074}
\pmmodifier{stevecheng}{10074}
\pmtitle{solving the Black-Scholes PDE by finite differences}
\pmrecord{6}{38694}
\pmprivacy{1}
\pmauthor{stevecheng}{10074}
\pmtype{Example}
\pmcomment{trigger rebuild}
\pmclassification{msc}{65M06}
\pmclassification{msc}{91B28}
\pmclassification{msc}{35K15}

% The standard font packages
\usepackage{amssymb}
\usepackage{amsmath}
\usepackage{amsfonts}

% For neatly defining theorems and definitions
%\usepackage{amsthm}

% Including EPS/PDF graphics (\includegraphics)
\usepackage{graphicx}

% Making matrix-based graphics
%%%\usepackage{xypic}

% Enumeration lists with different styles
%\usepackage{enumerate}

% Set up the theorem environments
%\newtheorem{thm}{Theorem}
%\newtheorem*{thm*}{Theorem}

% The standard number systems
\newcommand{\complex}{\mathbb{C}}
\newcommand{\real}{\mathbb{R}}
\newcommand{\rat}{\mathbb{Q}}
\newcommand{\nat}{\mathbb{N}}
\newcommand{\intset}{\mathbb{Z}}

% Absolute values and norms
% Normal, wide, and big versions of the delimeters
\providecommand{\abs}[1]{\lvert#1\rvert}
\providecommand{\absW}[1]{\left\lvert#1\right\rvert}
\providecommand{\absB}[1]{\Bigl\lvert#1\Bigr\rvert}
\providecommand{\norm}[1]{\lVert#1\rVert}
\providecommand{\normW}[1]{\left\lVert#1\right\rVert}
\providecommand{\normB}[1]{\Bigl\lVert#1\Bigr\rVert}

% Differentiation operators
\providecommand{\od}[2]{\frac{d #1}{d #2}}
\providecommand{\pd}[2]{\frac{\partial #1}{\partial #2}}
\providecommand{\pdd}[2]{\frac{\partial^2 #1}{\partial #2}}
\providecommand{\ipd}[2]{\partial #1 / \partial #2}

% Differentials on integrals
\newcommand{\dx}{\, dx}
\newcommand{\dt}{\, dt}
\newcommand{\dmu}{\, d\mu}

% Inner products
\providecommand{\ip}[2]{\langle {#1}, {#2} \rangle}

% Calligraphic letters
\newcommand{\sF}{\mathcal{F}}
\newcommand{\sD}{\mathcal{D}}
\newcommand{\Hilb}{\mathcal{H}}

% Operators and functions occassionally used in my articles
\DeclareMathOperator{\D}{D}
\DeclareMathOperator{\linspan}{span}
\DeclareMathOperator{\rank}{rank}
\DeclareMathOperator{\lindim}{dim}
\DeclareMathOperator{\sinc}{sinc}

\newcommand{\Dt}{\Delta t}
\newcommand{\Dx}{\Delta x}

\newcommand{\XL}{X_\mathrm{L}}
\newcommand{\XU}{X_\mathrm{U}}

\begin{document}
This entry presents some examples of solving
the Black-Scholes partial differential equation
in one space dimension:
\begin{align*}
rf = \pd{f}{t} + rx \pd{f}{x} + \tfrac12 \sigma^2 x^2 \pdd{f}{x^2}\,,
\quad f = f(t,x)\,,
\end{align*}
over the rectangle $0 \leq t \leq T$, $\XL \leq x \leq \XU$,
with various boundary conditions on the top, bottom, and right
sides of the rectangle.  The parameters $r$, $\sigma > 0$
are arbitrary constants.

(Add diagram of domain here...)

The partial differential equation can be solved
numerically using the basic methods
based on approximating the partial derivatives
with finite differences.

\subsection{Finite-difference formulae}

We summarize the equations for the finite differences below.

Let $n$, $m$, $k$ be some chosen positive integers,
which determine the grid on which we are approximating the solution
of the PDE.

Set $u_{i,j}$ to be the approximation to $f(T - i \Dt, j \Dx)$,
for $0 \leq i \leq m$ and $k \leq j \leq k+n+1$.
(For convenience, we have made time move ``backwards''
as we increase $i$, because the original PDE
is really a backwards heat equation, and evolves backwards in time.)

Also set:
\[
\Delta t = \frac{T}{m}\,, \quad \Delta x = \frac{\XU - \XL}{n+1}
\]


\textbf{Explicit method}.

\begin{align*}
\frac{u_{i+1, j} - u_{i, j}}{\Dt}
&= 
-r u_{i,j} + r j \Dx \, \frac{u_{i,j+1}- u_{i,j-1}}{2\Dx} \\
& \qquad
+ \frac12 \sigma^2 (j \Dx)^2 \, 
\frac{ u_{i,j-1} - 2u_{i,j} + u_{i,j+1}}{\Dx^2}
\end{align*}

\begin{align*}
u_{i+1, j} &= 
\Bigl( \tfrac12 (\sigma j)^2 \Dt - \tfrac12 rj \Dt \Bigr) \, u_{i,j-1} \\
& +
\Bigl( 1 - (\sigma j)^2 \Dt - r \Dt \Bigr) \, u_{i,j} \\
& +
\Bigl( \tfrac12 (\sigma j)^2 \Dt + \tfrac12 rj \Dt \Bigr) \, u_{i,j+1}\,.
\end{align*}

Since the PDE to solve is parabolic and time-dependent,
we can step through time to numerically approximate it.
Given $u_{i,*}$, we can recursively compute $u_{i+1,*}$.

(Add stencil of numerical method here...)

\textbf{Implicit method}.

\begin{align*}
\frac{u_{i+1, j} - u_{i, j}}{\Dt}
&= 
-r u_{i+1,j} + r j \Dx \, \frac{u_{i+1,j+1}- u_{i+1,j-1}}{2\Dx} \\
& \qquad
+ \frac12 \sigma^2 (j \Dx)^2 \, 
\frac{ u_{i+1,j-1} - 2u_{i+1,j} + u_{i+1,j+1}}{\Dx^2}
\end{align*}

\begin{align*}
u_{i, j} &= 
\Bigl( -\tfrac12 (\sigma j)^2 \Dt + \tfrac12 rj \Dt\Bigr) \, u_{i+1,j-1} \\
& +
\Bigl( 1 + (\sigma j)^2 \Dt + r \Dt \Bigr) \, u_{i+1,j} \\
& +
\Bigl( -\tfrac12 (\sigma j)^2 \Dt - \tfrac12 rj \Dt \Bigr) \, u_{i+1,j+1}\,.
\end{align*}

\textbf{Crank-Nicolson method}.

\begin{multline*}
 \Bigl( -\tfrac14 (\sigma j)^2 \Dt + \tfrac14 rj \Dt \Bigr) u_{i+1,j-1} \\
+ \Bigl( 1 + \tfrac12 (\sigma j)^2 \Dt - \tfrac12 r \Dt \Bigr) u_{i+1,j} 
+ \Bigl( -\tfrac14 (\sigma j)^2 \Dt - \tfrac14 rj \Dt \Bigr) u_{i+1,j+1} \\
= \Bigl( \tfrac14 (\sigma j)^2 \Dt - \tfrac14 rj \Dt \Bigr) u_{i,j-1} 
+ \Bigl( 1 - \tfrac12 (\sigma j)^2 \Dt - \tfrac12 r \Dt \Bigr) u_{i,j} \\
+ \Bigl( \tfrac14 (\sigma j)^2 \Dt + \tfrac14 rj \Dt \Bigr) u_{i,j+1}
\end{multline*}

\subsection{Convergence of methods}

(Briefly discuss convergence properties of these methods here...)

\subsection{Example results}

\begin{figure}
\includegraphics{european-call-surface.eps}
\caption{Basic stock call option price}
\end{figure}

Boundary conditions and parameters:

\begin{align*}
r = 0.10\,, \quad \sigma = 0.40\,, \quad T = 0.5\,, \quad K = 50.00\,.\\
\XL = 0\,, \quad \XU = 100.00\,. \\
\quad m = 100\,, \quad n = 200\,, \quad \Delta t = 0.005\,, \quad \Delta x = 0.4975\,.
\end{align*}

\begin{align*}
f(T, x) &= \max(x - K, \: 0)\,, & \XL \leq x \leq \XU \\
f(t, 0) &= 0\,, & 0 \leq t \leq T\,, \\
f(t, \XU) &= x - Ke^{-r(T-t)}\,, & 0 \leq t \leq T\,.
\end{align*}

(Describe analytic solution here...)

\begin{figure}
\includegraphics{up-and-out-call-surface.eps}
\caption{Price of call option with up-and-out barrier}
\end{figure}

\begin{align*}
r = 0.10\,, \quad \sigma = 0.40\,, \quad T = 0.5\,, \quad K = 50.00\,.\\
\XL = 0\,, \quad \XU = 100.00\,. \\
\quad m = 100\,, \quad n = 200\,, \quad \Delta t = 0.005\,, \quad \Delta x = 0.4975\,.
\end{align*}

\begin{align*}
f(T, x) &= \max(x - K, \: 0)\,, & \XL \leq x \leq \XU \\
f(t, 0) &= 0\,, & 0 \leq t \leq T\,, \\
f(t, \XU) &= 0\,, & 0 \leq t \leq T\,.
\end{align*}


\begin{itemize}
\item
\PMlinkexternal{Python program that implements the finite-difference methods for the above two problems, and plots the results}{http://svn.gold-saucer.org/math/PlanetMath/SolvingTheBlackScholesPDEByFiniteDifferences/bss.py}
\end{itemize}

%%%%%
%%%%%
\end{document}
