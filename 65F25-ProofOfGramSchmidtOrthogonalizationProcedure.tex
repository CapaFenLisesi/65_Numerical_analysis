\documentclass[12pt]{article}
\usepackage{pmmeta}
\pmcanonicalname{ProofOfGramSchmidtOrthogonalizationProcedure}
\pmcreated{2013-03-22 16:27:17}
\pmmodified{2013-03-22 16:27:17}
\pmowner{rspuzio}{6075}
\pmmodifier{rspuzio}{6075}
\pmtitle{proof of Gram-Schmidt orthogonalization procedure}
\pmrecord{11}{38611}
\pmprivacy{1}
\pmauthor{rspuzio}{6075}
\pmtype{Proof}
\pmcomment{trigger rebuild}
\pmclassification{msc}{65F25}
\pmsynonym{Gram-Schmidt}{ProofOfGramSchmidtOrthogonalizationProcedure}
\pmsynonym{orthogonalization}{ProofOfGramSchmidtOrthogonalizationProcedure}
%\pmkeywords{orthogonal}
%\pmkeywords{orthonormal}
%\pmkeywords{basis}
%\pmkeywords{inner product space}
%\pmkeywords{inner product}
\pmrelated{GramSchmidtOrthogonalization}
\pmrelated{ExampleOfGramSchmidtOrthogonalization}
\pmrelated{InnerProductSpace}
\pmrelated{InnerProduct}
\pmrelated{QRDecomposition}
\pmrelated{NormedVectorSpace}
\pmrelated{Orthogonal}
\pmrelated{OrthogonalVectors}
\pmrelated{Basis}
\pmrelated{Span}
\pmrelated{LinearIndependence}
\pmrelated{Orthonormal}
\pmrelated{OrthonormalBasis}
\pmdefines{Gram-Schmidt orthogonalization}

\endmetadata

% this is the default PlanetMath preamble.  as your knowledge
% of TeX increases, you will probably want to edit this, but
% it should be fine as is for beginners.

% almost certainly you want these
\usepackage{amssymb}
\usepackage{amsmath}
\usepackage{amsfonts}
\usepackage{amsthm}

% used for TeXing text within eps files
%\usepackage{psfrag}
% need this for including graphics (\includegraphics)
%\usepackage{graphicx}
% for neatly defining theorems and propositions
%\usepackage{amsthm}
% making logically defined graphics
%%%\usepackage{xypic}

% there are many more packages, add them here as you need them

% define commands here
\theoremstyle{plain}
\newtheorem*{thm}{Theorem}
\newtheorem*{lem}{Lemma}
\newtheorem*{cor}{Corollary}

\DeclareMathOperator{\spn}{Span}


\begin{document}
Note that, while we state the following as a theorem for the sake of logical completeness and to establish notation, our definition of Gram-Schmidt orthogonalization is wholly equivalent to that given in the defining entry.
\begin{thm}
\emph{(Gram-Schmidt Orthogonalization)}
Let $\{\mathbf{u_k}\}_{k=1}^n$ be a basis for an inner product space $V$ with inner product $\big<,\big>$. Define $\mathbf{v_1}=\tfrac{\mathbf{u_1}}{\mid\mid\mathbf{u_1}\mid\mid}$ and $\{\mathbf{m_k}\}_{k=2}^n$ recursively by 
\begin{equation}
\mathbf{m_k}=\mathbf{u_k}-\big<\mathbf{u_k},\mathbf{v_1}\big>\mathbf{v_1}-
\big<\mathbf{u_k},\mathbf{v_2}\big>\mathbf{v_2}-\ldots
-\big<\mathbf{u_k},\mathbf{v_{k-1}}\big>\mathbf{v_{k-1}}\text{,}
\end{equation}
where $\mathbf{v_k}=\tfrac{\mathbf{m_k}}{\mid\mid\mathbf{m_k}\mid\mid}$ for $2\leq k\leq n$. Then $\{\mathbf{v_k}\}_{k=1}^n$ is an orthonormal basis for $V$. 
\end{thm}
\begin{proof}
We proceed by induction on $n$. In the case $n=1$, we suppose 
$\{\mathbf{u_k}\}_{k=1}^n=\{\mathbf{u_k}\}_{k=1}^1=\mathbf{u_1}$ is a basis for the inner product space $V$. Letting $\mathbf{v_1}=\tfrac{\mathbf{u_1}}{\mid\mid\mathbf{u_1}\mid\mid}$, it is clear that $\mathbf{v_1}\in \spn\big(\mathbf{u_1}\big)$, whence it follows that $\spn\big(\mathbf{v_1}\big)=\spn\big(\mathbf{u_1}\big)=V$. Thus $\mathbf{v_1}$ is an orthonormal basis for $V$, and the result holds for $n=1$. Now let $n\geq 1\in\mathbb{N}$, and suppose the result holds for arbitrary $n$. Let $\{\mathbf{u_k}\}_{k=1}^{n+1}$ be a basis for an inner product space $V$. By the inductive hypothesis we may use $\{\mathbf{u_k}\}_{k=1}^n$ to construct an orthonormal set of vectors $\{\mathbf{v_k}\}_{k=1}^n$ such that $\spn\big(\{\mathbf{v_k}\}_{k=1}^n\big)=\spn\big(\{\mathbf{u_k}\}_{k=1}^n\big)$. In accordance with the procedure outlined in the statement of the theorem, let $\mathbf{m_{n+1}}$ be defined as
\begin{equation*}
\mathbf{u_{n+1}}-
\big<\mathbf{u_{n+1}},\mathbf{v_1}\big>\mathbf{v_1}
-\big<\mathbf{u_{n+1}},\mathbf{v_2}\big>\mathbf{v_2}
-\ldots
-\big<\mathbf{u_{n+1}},\mathbf{v_n}\big>\mathbf{v_n}
=\mathbf{u_{n+1}}-\sum\limits_{i=1}^n \big<\mathbf{u_{n+1}},\mathbf{v_i}\big>\mathbf{v_i}\text{.}
\end{equation*}
First we show that the vectors $\mathbf{v_1},\mathbf{v_2},\ldots,\mathbf{v_n},\mathbf{m_{n+1}}$ are mutually orthogonal.
Consider the inner product of 
$\mathbf{m_{n+1}}$ with $\mathbf{v_j}$ for $1\leq j\leq n$. By construction, we have
\begin{equation*}
\big<\mathbf{m_{n+1}},\mathbf{v_j}\big>=\biggl<\mathbf{u_{n+1}}
-\sum_{i=1}^n\big<\mathbf{u_{n+1}},
\mathbf{v_i}\big>\mathbf{v_i},\mathbf{v_j}\biggr>
=\big<\mathbf{u_{n+1}},\mathbf{v_j}\big>
-\biggl<\sum\limits_{i=1}^n\big<\mathbf{u_{n+1}},\mathbf{v_i}\big>
\mathbf{v_i},\mathbf{v_j}\biggr>\text{.}
\end{equation*}
Now since $\{\mathbf{v_k}\}_{k=1}^n$ is an orthonormal set of vectors, whence
$\big<\mathbf{v_i},\mathbf{v_j}\big>=\delta_{ij}$, each term in the summation on the right-hand side of the preceding equation will vanish except for the term where $i=j$. Thus by this and the preceding equation, we have
\begin{multline*}
\big<\mathbf{m_{n+1}},\mathbf{v_j}\big>
=\big<\mathbf{u_{n+1}},\mathbf{v_j}\big>
-\biggl<\sum\limits_{i=1}^n\big<\mathbf{u_{n+1}},\mathbf{v_i}\big>
\mathbf{v_i},\mathbf{v_j}\biggr>
=\big<\mathbf{u_{n+1}},\mathbf{v_j}\big>
-\biggl<\big<\mathbf{u_{n+1}},\mathbf{v_j}\big>\mathbf{v_j},\mathbf{v_j}\biggr>\\
=\big<\mathbf{u_{n+1}},\mathbf{v_j}\big>
-\big<\mathbf{u_{n+1}},\mathbf{v_{j}}\big>\big<\mathbf{v_j},\mathbf{v_j}\big>
=\big<\mathbf{u_{n+1}},\mathbf{v_{j}}\big>
-\big<\mathbf{u_{n+1}},\mathbf{v_{j}}\big>
=0\text{.}
\end{multline*}
Thus $\mathbf{m_{n+1}}$ is orthogonal to $\mathbf{v_j}$ for $1\leq j\leq n$, so we may take $\mathbf{v_{n+1}}=\tfrac{\mathbf{m_{n+1}}}{\mid\mid\mathbf{m_{n+1}}\mid\mid}$ to have $\{\mathbf{v_k}\}_{k=1}^{n+1}$ an orthonormal set of vectors. Finally we show that $\{\mathbf{v_k}\}_{k=1}^{n+1}$ is a basis for $V$. 
By construction, each $\mathbf{v_k}$ is a linear combination of the vectors $\{\mathbf{u_k}\}_{k=1}^{n+1}$, so we have $n+1$ orthogonal, hence linearly independent vectors in the $n+1$ dimensional space $V$, from which it follows that $\{\mathbf{v_k}\}_{k=1}^{n+1}$ is a basis for $V$. Thus the result holds for $n+1$, and by the principle of induction, for all $n$. 
\end{proof}
%%%%%
%%%%%
\end{document}
