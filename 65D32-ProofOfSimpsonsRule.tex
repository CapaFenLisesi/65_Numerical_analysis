\documentclass[12pt]{article}
\usepackage{pmmeta}
\pmcanonicalname{ProofOfSimpsonsRule}
\pmcreated{2013-03-22 14:50:25}
\pmmodified{2013-03-22 14:50:25}
\pmowner{drini}{3}
\pmmodifier{drini}{3}
\pmtitle{proof of Simpson's rule}
\pmrecord{4}{36509}
\pmprivacy{1}
\pmauthor{drini}{3}
\pmtype{Proof}
\pmcomment{trigger rebuild}
\pmclassification{msc}{65D32}
\pmclassification{msc}{41A55}
\pmclassification{msc}{26A06}
\pmclassification{msc}{28-00}

\endmetadata

\usepackage{graphicx}
%%%\usepackage{xypic} 
\usepackage{bbm}
\newcommand{\Z}{\mathbbmss{Z}}
\newcommand{\C}{\mathbbmss{C}}
\newcommand{\R}{\mathbbmss{R}}
\newcommand{\Q}{\mathbbmss{Q}}
\newcommand{\mathbb}[1]{\mathbbmss{#1}}
\newcommand{\figura}[1]{\begin{center}\includegraphics{#1}\end{center}}
\newcommand{\figuraex}[2]{\begin{center}\includegraphics[#2]{#1}\end{center}}
\newtheorem{dfn}{Definition}
\begin{document}
We want to derive Simpson's rule for 
\[
\int_a^b f(x) \,dx.
\]

We will use Newton and Cotes formulas for $n=2$. In this case, $x_0=a$, $x_2=b$ and $x_1 = (a+b)/2$. We use Lagrange's interpolation formula to get a polynomial $p(x)$ such that $p(x_j)=f(x_j)$ for $j=0,1,2$.

The corresponding interpolating polynomial is
\[
p(x)=f(x_1)\frac{(x-x_2)(x-x_3)}{(x_1-x_2)(x_1-x_3)}+
f(x_2)\frac{(x-x_1)(x-x_3)}{(x_2-x_1)(x_2-x_3)}
+f(x_3)\frac{(x-x_1)(x-x_2)}{(x_3-x_1)(x_3-x_2)}.
\]
and thus
\[
\int_a^b f(x) \,dx\approx 
\int_a^b f(x_1)\frac{(x-x_2)(x-x_3)}{(x_1-x_2)(x_1-x_3)}+
f(x_2)\frac{(x-x_1)(x-x_3)}{(x_2-x_1)(x_2-x_3)}
+f(x_3)\frac{(x-x_1)(x-x_2)}{(x_3-x_1)(x_3-x_2)}\,dx.
\]

Since integration is linear, we are concerned only with integrating each term in the sum. Now, taking $x_j = a + hj$ where $j=0,1,2$ and $h=|b-a|/2$, we can rewrite the quotients on the last integral as
\[
\int_a^b p(x)\, dx = hf(x_0)\int_0^2\frac{(t-1)(t-2)}{(0-1)(0-2)}\,dt + 
 hf(x_1)\int_0^2\frac{(t-0)(t-2)}{(1-0)(1-2)}\,dt + 
 hf(x_2)\int_0^2\frac{(t-0)(t-1)}{(2-0)(2-1)}\,dt.
\]
and if we calculate the integrals on the last expression we get
\[
\int_a^b p(x)\,dx=hf(x_0)\frac{1}{3} + hf(x_1)\frac{4}{3}+hf(x_2)\frac{1}{3},
\]
which is Simpson's rule:
\[
\int_a^b f(x)\,dx \approx \frac{h}{3}(f(x_0) + 4f(x_1) + f(x_2)).
\]
%%%%%
%%%%%
\end{document}
