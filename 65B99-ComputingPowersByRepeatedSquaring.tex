\documentclass[12pt]{article}
\usepackage{pmmeta}
\pmcanonicalname{ComputingPowersByRepeatedSquaring}
\pmcreated{2013-03-22 19:18:45}
\pmmodified{2013-03-22 19:18:45}
\pmowner{rspuzio}{6075}
\pmmodifier{rspuzio}{6075}
\pmtitle{computing powers by repeated squaring}
\pmrecord{8}{42250}
\pmprivacy{1}
\pmauthor{rspuzio}{6075}
\pmtype{Algorithm}
\pmcomment{trigger rebuild}
\pmclassification{msc}{65B99}

\endmetadata

% this is the default PlanetMath preamble.  as your knowledge
% of TeX increases, you will probably want to edit this, but
% it should be fine as is for beginners.

% almost certainly you want these
\usepackage{amssymb}
\usepackage{amsmath}
\usepackage{amsfonts}

% used for TeXing text within eps files
%\usepackage{psfrag}
% need this for including graphics (\includegraphics)
%\usepackage{graphicx}
% for neatly defining theorems and propositions
%\usepackage{amsthm}
% making logically defined graphics
%%%\usepackage{xypic}

% there are many more packages, add them here as you need them

% define commands here

\begin{document}
From time to time, there arise occasions where one wants to raise a
number to a large integer power.  While one can compute $x^n$ by
multiplying $x$ by itself $n$ times, not only does this entail a
large expenditure of effort when $n$ is large, but roundoff errors
can accumulate.  A more efficient approach is to first rise $x$ to
powers of two by successive squaring, then multiply to obtain $x^n$.
This procedure will now be explained by computing $1.08145^{187}$.

The first step is to express $n$ as a sum of powers of two, i.e. in
base 2.  In our example, we have $187 = 1 + 2 + 8 + 16 + 32 + 128$.

By the basic properties of exponentiation, we therefore have
$x^{187} = x \cdot x^2 \cdot x^8 \cdot x^{16} \cdot x^{32} \cdot x^{128}$.

The next step is to raise 1.08145 to powers of two, which we accomplish
by repeatedly squaring like so:  (In order to guard against roundoff
error, we work with more decimal places than needed and round off at the 
end of the computation.)
\begin{align*}
x &= 1.08145 \\
x^2 &= 1.1695341 \\
x^4 &= (x^2)^2 = 1.1695341^2 = 1.3678100 \\
x^8 &= (x^4)^2 = 1.3678100^2 = 1.8709042 \\
x^{16} &= (x^8)^2 = 1.8709042^2 = 3.5002825 \\
x^{32} &= (x^{16})^2 = 3.5002825^2 = 12.251978\\
x^{64} &= (x^{32})^2 = 12.251978^2 = 150.110965\\
x^{128} &= (x^{64})^2 = 150.110965^2 = 26523642.95\\
\end{align*}

Finally, we multiply together the appropriate powers to obtain our
answer:
\[
 1.08145 \cdot 1.1695341 \cdot 1.8709042 \cdot 3.5002825 \cdot
 12.251978 \cdot 26523642.95 = 2691617615
\]

Hence, we compute $1.08145^{187} = 2.69162 \times 10^9$.

Note that this computation involved only 12 multiplications
as opposed to 186 multiplications.   More generally, we will
require $\log_2 n$ repeated squarings and up to $\log_2 n$
multiplications of the repeated squares, or a total of between
$\log_2 n$ and $2 \log_2 n$ multiplications to compute
$x^n$ this way.

More generally, the same method can be used to compute $x^n$
when $x$ is a complex number or a matrix, or something even
more general, such as an element of a group.  Thus this
approach can be adapted to computing exponentials and 
trigonometric functions, numerical approximation of differential
equation, and the like.

%%%%%
%%%%%
\end{document}
