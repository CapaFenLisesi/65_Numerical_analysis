\documentclass[12pt]{article}
\usepackage{pmmeta}
\pmcanonicalname{EulerMaclaurinSummationFormula}
\pmcreated{2013-03-22 11:46:01}
\pmmodified{2013-03-22 11:46:01}
\pmowner{KimJ}{5}
\pmmodifier{KimJ}{5}
\pmtitle{Euler-Maclaurin summation formula}
\pmrecord{9}{30220}
\pmprivacy{1}
\pmauthor{KimJ}{5}
\pmtype{Theorem}
\pmcomment{trigger rebuild}
\pmclassification{msc}{65B15}
\pmclassification{msc}{00-02}
%\pmkeywords{number theory}
\pmrelated{BernoulliNumber}

\endmetadata

\usepackage{amssymb}
\usepackage{amsmath}
\usepackage{amsfonts}
\usepackage{graphicx}
%%%%\usepackage{xypic}
\begin{document}
Let $B_r$ be the $r\mbox{th}$ Bernoulli number, and $B_r(x)$ be the $r\mbox{th}$ Bernoulli periodic function. For any integer $k \geq 0$ and for any function $f$ of class $C^{k+1}$ on $[a,b],a,b \in \mathbb{Z}$, we have
\[
\sum_{a < n \leq b} f(n) = \int_a^b f(t)dt + \sum_{r=0}^k \frac{(-1)^{r+1}B_{r+1}}{(r+1)!}(f^{(r)}(b) - f^{(r)}(a)) + \frac{(-1)^k}{(k+1)!} \int_a^b B_{k+1}(t)f^{(k+1)}(t)dt. \]
%%%%%
%%%%%
%%%%%
%%%%%
\end{document}
