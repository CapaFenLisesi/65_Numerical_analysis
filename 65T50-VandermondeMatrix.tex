\documentclass[12pt]{article}
\usepackage{pmmeta}
\pmcanonicalname{VandermondeMatrix}
\pmcreated{2013-03-22 13:04:19}
\pmmodified{2013-03-22 13:04:19}
\pmowner{akrowne}{2}
\pmmodifier{akrowne}{2}
\pmtitle{Vandermonde matrix}
\pmrecord{4}{33481}
\pmprivacy{1}
\pmauthor{akrowne}{2}
\pmtype{Definition}
\pmcomment{trigger rebuild}
\pmclassification{msc}{65T50}
\pmclassification{msc}{65F99}
\pmclassification{msc}{15A57}

\endmetadata

\usepackage{amssymb}
\usepackage{amsmath}
\usepackage{amsfonts}

%\usepackage{psfrag}
%\usepackage{graphicx}
%%%\usepackage{xypic}
\begin{document}
A \emph{Vandermonde matrix} is any $(n+1)\times(n+1)$ matrix of the form 

$$ \begin{bmatrix}
 1 & x_0 & x_0^2 & \cdots & x_0^n   \\
 1 & x_1 & x_1^2 & \cdots & x_1^n   \\
 \vdots & \vdots & \vdots & \ddots & \vdots \\
 1 & x_n & x_n^2 & \cdots & x_n^n 
\end{bmatrix} $$

Vandermonde matrices usually arise when considering systems of polynomials evaluated at specific points (i.e. in interpolation or approximation).  This may happen, for example, when trying to solve for constants from initial conditions in systems of differential equations or recurrence relations.

Vandermonde matrices also appear in the computation of FFTs (Fast Fourier Transforms).  Here the fact that Vandermonde systems $Vz = b$ can be solved in $\mathcal{O}(n \log n)$ flops by taking advantage of their \PMlinkescapetext{recursive block structure} comes into play.

\subsection{References}
\begin{enumerate}
\item Golub and Van Loan, \emph{Matrix Computations}, Johns Hopkins University Press 1993
\end{enumerate}
%%%%%
%%%%%
\end{document}
