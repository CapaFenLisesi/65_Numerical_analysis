\documentclass[12pt]{article}
\usepackage{pmmeta}
\pmcanonicalname{LinearInterpolation}
\pmcreated{2013-03-22 17:28:32}
\pmmodified{2013-03-22 17:28:32}
\pmowner{CWoo}{3771}
\pmmodifier{CWoo}{3771}
\pmtitle{linear interpolation}
\pmrecord{13}{39861}
\pmprivacy{1}
\pmauthor{CWoo}{3771}
\pmtype{Definition}
\pmcomment{trigger rebuild}
\pmclassification{msc}{65D05}
\pmclassification{msc}{41A05}

\endmetadata

\usepackage{amssymb,amscd}
\usepackage{amsmath}
\usepackage{amsfonts}
\usepackage{mathrsfs}
\usepackage{color}

% used for TeXing text within eps files
%\usepackage{psfrag}
% need this for including graphics (\includegraphics)
%\usepackage{graphicx}
% for neatly defining theorems and propositions
\usepackage{amsthm}
% making logically defined graphics
%%\usepackage{xypic}
\usepackage{pst-plot}
\usepackage{psfrag}

% define commands here
\newtheorem{prop}{Proposition}
\newtheorem{thm}{Theorem}
\newtheorem{ex}{Example}
\newcommand{\real}{\mathbb{R}}
\newcommand{\pdiff}[2]{\frac{\partial #1}{\partial #2}}
\newcommand{\mpdiff}[3]{\frac{\partial^#1 #2}{\partial #3^#1}}
\begin{document}
Among the many interpolation techniques that are available, \emph{linear interpolation} is one of the easiest to understand and implement, as the interpolating function is pieced together by a series of line segments connecting the breakpoints.

Suppose we have a finite set $S$ of ordered pairs $(x_1,y_1),\ldots,(x_n,y_n)$ of real numbers such that $x_1<x_2<\cdots <x_n$.  The \emph{linear interpolation function} of $S$ is a real-valued function $f$ defined on $[x_1,x_n]$ such that, for $i=1,\ldots,n-1$, 
$$f(x)=y_i + m_i (x-x_i),\quad\mbox{ where }m_i=\frac{y_{i+1}-y_i}{x_{i+1}-x_i}\mbox{ and }x\in [x_i,x_{i+1}].$$
In other words, $f$ is a piecewise linear function such that $f$ is linear in each of the interval $[x_i,x_{i+1}]$ for $i=1,\ldots,n-1$.  When the points (in $S$) belong to the graph of a function $g$ defined on a subset of $[x_1,x_n]$, we say that $f$ interpolates $g$.  We also say that $f$ interpolates $S$, as $S$ can be viewed as the graph of the function $g_S$ defined on $\lbrace x_1,\ldots, x_n\rbrace$ such that $g_S(x_i)=y_i$.

Visually, the interpolation function can be constructed by line segments whose end points are pairs of points $(x_i,y_i)$ and $(x_{i+1},y_{i+1})$ for each $i=1,\ldots,n-1$.  The follow graph shows the linear interpolation function $f$ (in blue) of a set consisting of seven points (in dark green).  Note that $f$ interpolates any function $g$ defined on a subset of $[x_1,x_n]$ such that $g(x_i)=y_i$.

\begin{center}
\begin{pspicture}(-7,-1.5)(7,3.5)
\psset{unit=0.8cm}
\rput[l](-7,0){.}
\rput[r](7,0){.}
\rput[a](3,-1){.}
\rput[b](-3.7,3.1){.}
\psaxes[Dx=10,Dy=10]{<->}(-2,0)(-7,-1.5)(7,3.5)
\pscurve[linecolor=red]{<->}(-6,1)(-4,3)(-3,3)(0,0.5)(1,1)(3,-1)(6,2)
\rput[b](6.74,-0.4){$x$}
\rput[b](-1.75,3.2){$y$}
\rput[b](4.75,-1){\textcolor{red}{$g(x)$}}
\rput[b](4.5,1){\textcolor{blue}{$f(x)$}}
\psline[linecolor=blue]{-}(-6,1)(-4,3)(-3,3)(0,0.5)(1,1)(3,-1)(6,2)
\newrgbcolor{darkgreen}{0 0.75 0}
\psdots[linecolor=darkgreen,dotsize=5pt](-6,1)(-4,3)(-3,3)(0,0.5)(1,1)(3,-1)(6,2)
\end{pspicture}
\end{center}

\textbf{Example.}  Interpolate $\lbrace (4,7),(2,3),(6,1)\rbrace$ using linear interpolation.

Arrange the points so the $x$-coordinates are in the ascending order.  There are two line segments associated with these three points: $\ell_1$ with end points $(2,3),(4,7)$ and $\ell_2$ with end points $(4,7),(6,1)$.  Next, calculate the slopes with respect to each line segments: $$m_1=\frac{7-3}{4-2}=2\qquad\mbox{ and }\qquad m_2=\frac{1-7}{6-4}=-3.$$
Therefore, the linear interpolation function $f$ is given by
\begin{displaymath}
f(x) = \left\{
\begin{array}{ll}
3+2(x-2) = 2x - 1 & \textrm{if }x\in [2,4]\\
7+(-3)(x-4)= -3x+19 & \textrm{if }x\in [4,6].
\end{array}
\right.
\end{displaymath}
%%%%%
%%%%%
\end{document}
