\documentclass[12pt]{article}
\usepackage{pmmeta}
\pmcanonicalname{Rounding}
\pmcreated{2013-03-22 17:27:27}
\pmmodified{2013-03-22 17:27:27}
\pmowner{CWoo}{3771}
\pmmodifier{CWoo}{3771}
\pmtitle{rounding}
\pmrecord{10}{39840}
\pmprivacy{1}
\pmauthor{CWoo}{3771}
\pmtype{Definition}
\pmcomment{trigger rebuild}
\pmclassification{msc}{65G99}
\pmclassification{msc}{65D99}
\pmclassification{msc}{00A69}
\pmclassification{msc}{65G50}
\pmsynonym{round up}{Rounding}
\pmsynonym{round down}{Rounding}
\pmsynonym{round to}{Rounding}
\pmdefines{rounding up}
\pmdefines{rounding down}
\pmdefines{symmetric arithmetic rounding}
\pmdefines{rounding error}
\pmdefines{truncation}
\pmdefines{rounded to}
\pmdefines{banker's rounding}

\endmetadata

\usepackage{amssymb,amscd}
\usepackage{amsmath}
\usepackage{amsfonts}
\usepackage{mathrsfs}

% used for TeXing text within eps files
%\usepackage{psfrag}
% need this for including graphics (\includegraphics)
%\usepackage{graphicx}
% for neatly defining theorems and propositions
\usepackage{amsthm}
% making logically defined graphics
%%\usepackage{xypic}
\usepackage{pst-plot}
\usepackage{psfrag}

% define commands here
\newtheorem{prop}{Proposition}
\newtheorem{thm}{Theorem}
\newtheorem{ex}{Example}
\newcommand{\real}{\mathbb{R}}
\newcommand{\pdiff}[2]{\frac{\partial #1}{\partial #2}}
\newcommand{\mpdiff}[3]{\frac{\partial^#1 #2}{\partial #3^#1}}
\begin{document}
\emph{Rounding} is a general technique for approximating a real number by a decimal fraction.  There are several ways of rounding a real number, five of which are the most common: \emph{rounding up}, \emph{rounding down}, \emph{truncation}, \emph{ordinary rounding} (or \emph{rounding} for short), and \emph{banker's rounding}.

\subsubsection*{Rounding to an Integer}

The simplest kind of rounding is that of rounding a real number to an integer.  Let $r$ be a real number.  Then
\begin{description}
\item[rounding up:] rounding up of $r$ is taking the smallest integer that is greater than or equal to $r$.  This integer is denoted by the ceiling function $$\lceil r \rceil:=\min\lbrace n\in \mathbb{Z}\mid n\ge r \rbrace.$$  Examples: $\lceil 2.1 \rceil = 3$, and $\lceil 62.672 \rceil=63$.
\item[rounding down:] rounding down of $r$ is taking the largest integer that is less than or equal to $r$.  This integer is denoted by the floor function 
\begin{displaymath}
\lfloor r \rfloor:=\max\lbrace n\in \mathbb{Z}\mid n\le r\rbrace = \left\{
\begin{array}{ll}
\lceil r \rceil & \textrm{if $r$ is an integer}\\
\lceil r\rceil - 1 & \textrm{otherwise.}
\end{array}
\right.
\end{displaymath}
Examples: $\lfloor 1.24 \rfloor = 1$, and $\lfloor -2.63 \rfloor = -3$.  
\item[truncation:] rounding by truncation is done by ignoring all decimals to the right of the decimal point, which is equivalent to taking only the integer part of $r$.  The truncation of $r$ is denoted by $[r]$.  In terms of rounding up and rounding down: we have
\begin{displaymath}
[r] = \left\{
\begin{array}{ll}
\lfloor r \rfloor & \textrm{if $r\ge 0$}\\
\lceil r \rceil & \textrm{if $r< 0.$}
\end{array}
\right.
\end{displaymath}
If we write $r$ as a decimal number using decimal expansion, then $[r]$ is the integral portion of $r$.

Examples: $[2.354]=2$, and $[-81.67]=-81$.
\item[ordinary rounding:] this is the most commonly used of the rounding methods described so far.  (Ordinary) rounding of $r$ is finding the closest integer to $r$, and if $r$ is exactly half way between two integers, use the larger of the two as the result.  Let $R(r)$ represents the ordinary rounding of $r$.  It is easy to see that $$R(r)=\lfloor r+0.5 \rfloor.$$  Examples: $R(-3.37)=-3$, while $R(7.5)=8$.

There is an easy algorithm of rounding $r$ to the nearest integer.
\begin{enumerate}
\item write $r$ as a decimal number using decimal expansion
\item if the tenths decimal place value is less than $5$, then $R(r)=[r]$
\item if the tenths decimal place value is at least $5$, then $R(r)=[r]+1$.
\end{enumerate}
\item[banker's rounding:] a variant of the ordinary rounding is the banker's rounding: if $r$ is exactly half way between two integers, and the integer portion of $r$ is even, round down $r$.  Otherwise, use ordinary rounding on $r$.  If $B(r)$ denotes the banker's rounding of $r$, then it can be defined as
\begin{displaymath}
B(r) = \left\{
\begin{array}{ll}
\lfloor r \rfloor & \textrm{if $[r]$ is even, and $2r\in \mathbb{Z}$}\\
R(r) & \textrm{otherwise.}
\end{array}
\right.
\end{displaymath}
For example, $B(3.5)=4$, while $B(2.5)=2$.

\item[stochastic rounding:] this rounding method requires the aid of a random number generator.  Rounding of $r$ may be done using any of the above methods when $r$ is not exactly half way between two consecutive integers.  Otherwise, $r$ is randomly rounded up or down based on the outcome of randomly selecting a number between 0 and 1 using a random number generator.   The choice of rounding up (and thus down) depends on how numbers are in $[0,1]$ are allocated for rounding up (or down).

\item[alternate rounding:] this rounding method, like the last one, uses other available methods except when the number in question $r$ is exactly half way between two consecutive integers.  However, this method is used in a situation where a sequence of numbers needs to be rounded:  
\begin{enumerate}
\item
the first number in the sequence is rounded using any of the above methods;
\item
when the $n$-th number is rounded, the $(n+1)$-th number is rounded as follows: if the number is exactly half way between two consecutive integers, then it is rounded down if the $n$-th number is rounded up, and vice versa.  Otherwise, use the rounding method used to round the first number in the sequence.
\end{enumerate}
\end{description}

\subsubsection*{Rounding to a Decimal Fraction}

More generally, the three methods described can be applied to rounding of $r$ to a decimal fraction.  The general procedure is as follows:
\begin{enumerate}
\item
First, specify how accurately we want to round $r$.  This can be accomplished by specifying to what decimal place we want to approximate $r$.  Let this place be $n$ (note that $n>0$ if it is to the right of the decimal point and $n<0$ otherwise).
\item
Write $r$ as a decimal number using decimal expansion.
\item
Multiply $r$ by $10^n$.  By doing this, we are basically moving the decimal point so it is positioned between the $n$-th decimal place and the $(n+1)$-th decimal place.
\item
Use any of the four methods above to round $10^nr$.
\item
Divide the rounded number by $10^n$ to get the result.
\end{enumerate}

In practice, steps $3$ through $5$ can be combined into one step, simply by performing the rounding operation at the specified decimal place as if it were the ones place.  For example, rounding $\pi=3.14159...$ to the nearest thousandths place is $3.142$, the thousandths place value $1$ is increased to $2$ because the ten thousandths place is $5$.

\textbf{Remark}.  In general, rounding to the $n$-th decimal place can be thought of as a function $f$ from $\mathbb{R}$ to $D$, the set of all decimal fractions, such that 
\begin{itemize}
\item $|f(r)-r|\le 10^n$, and
\item $f(r)=r$ if $10^nr\in \mathbb{Z}$.
\end{itemize}
If $g:\mathbb{R}\to \mathbb{Z}$ denotes any of the four rounding methods described in the previous section, and $g_n$ corresponds to rounding to the $n$-th decimal place using method $g$ in step $4$ above, then the entire rounding process can be summarized by a single formula: $$g_n(r)=\frac{g(10^n r)}{10^n}.$$
%%%%%
%%%%%
\end{document}
