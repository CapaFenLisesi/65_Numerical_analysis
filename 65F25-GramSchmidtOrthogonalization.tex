\documentclass[12pt]{article}
\usepackage{pmmeta}
\pmcanonicalname{GramSchmidtOrthogonalization}
\pmcreated{2013-03-22 12:06:14}
\pmmodified{2013-03-22 12:06:14}
\pmowner{akrowne}{2}
\pmmodifier{akrowne}{2}
\pmtitle{Gram-Schmidt orthogonalization}
\pmrecord{9}{31216}
\pmprivacy{1}
\pmauthor{akrowne}{2}
\pmtype{Algorithm}
\pmcomment{trigger rebuild}
\pmclassification{msc}{65F25}
\pmsynonym{Gram-Schmidt decomposition}{GramSchmidtOrthogonalization}
\pmsynonym{Gram-Schmidt orthonormalization}{GramSchmidtOrthogonalization}
\pmsynonym{Gram-Schmidt process}{GramSchmidtOrthogonalization}
\pmrelated{HouseholderTransformation}
\pmrelated{GivensRotation}
\pmrelated{QRDecomposition}
\pmrelated{AnExampleForSchurDecomposition}

\usepackage{amssymb}
\usepackage{amsmath}
\usepackage{amsfonts}
\usepackage{graphicx}
%%%\usepackage{xypic}
\begin{document}
Any set of linearly independent vectors $v_1,\ldots,v_n$ can be converted into a set of orthogonal vectors $q_1,\ldots,q_n$ by the Gram-Schmidt process. In three dimensions, $v_1$ determines a line; the vectors $v_1$ and $v_2$ determine a plane. The vector $q_1$ is the unit vector in the direction $v_1$. The (unit) vector $q_2$ lies in the plane of $v_1, v_2$, and is normal to $v_1$ (on the same side as $v_2$. The (unit) vector $q_3$ is normal to the plane of $v_1, v_2$, on the same side as $v_3$, etc.

In general, first set $u_1 = v_1$, and then each $u_i$ is made orthogonal to the preceding $u_1,\ldots u_{i-1}$ by subtraction of the projections of $v_i$ in the directions of $u_1,\ldots,u_{i-1}$ :

$$ u_i = v_i - \sum_{j=1}^{i-1} \frac{u_j^Tv_i}{u_j^Tu_j} u_j$$

The $i$ vectors $u_i$ span the same subspace as the $v_i$. The vectors $q_i=u_i/||u_i||$ are orthonormal. This leads to the following theorem:

{\bf Theorem.}

Any $m \times n$ matrix $A$ with linearly independent columns can be factorized into a product, $A = QR$. The columns of $Q$ are orthonormal and $R$ is upper triangular and invertible.

This ``classical'' Gram-Schmidt method is often numerically unstable, see [Golub89] for a ``modified'' Gram-Schmidt method.

{\bf References}

\begin{itemize}
\item Originally from The Data Analysis Briefbook
(\PMlinkexternal{http://rkb.home.cern.ch/rkb/titleA.html}{http://rkb.home.cern.ch/rkb/titleA.html})
\end{itemize}

\begin{itemize}
\item[Golub89] Gene H. Golub and Charles F. van Loan: Matrix Computations, 2nd edn., The John Hopkins University Press, 1989.
\end{itemize}
%%%%%
%%%%%
%%%%%
\end{document}
