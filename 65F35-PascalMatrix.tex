\documentclass[12pt]{article}
\usepackage{pmmeta}
\pmcanonicalname{PascalMatrix}
\pmcreated{2013-03-22 13:44:54}
\pmmodified{2013-03-22 13:44:54}
\pmowner{bbukh}{348}
\pmmodifier{bbukh}{348}
\pmtitle{Pascal matrix}
\pmrecord{6}{34445}
\pmprivacy{1}
\pmauthor{bbukh}{348}
\pmtype{Definition}
\pmcomment{trigger rebuild}
\pmclassification{msc}{65F35}
\pmclassification{msc}{15A12}
\pmclassification{msc}{15A09}
\pmclassification{msc}{15A57}

\endmetadata

% almost certainly you want these
\usepackage{amssymb}
\usepackage{amsmath}
\usepackage{amsfonts}

%%%\usepackage{xypic}

\makeatletter
\@ifundefined{bibname}{}{\renewcommand{\bibname}{References}}
\makeatother
\begin{document}
{\bf Definition} The \emph{Pascal matrix} $P$ of order $n$ is the real 
square $n\times n$ matrix whose entries are \cite{higham}
 $$ P_{ij} = { i+j-2 \choose j-1}. $$
 
For $n=5$, 
 $$P= \begin{pmatrix}
1 & 1 & 1 & 1  & 1\\
1 & 2 & 3 & 4  & 5 \\
1 & 3 & 6 & 10  & 15\\
1 & 4 & 10 & 20 & 35 \\
1 & 5 & 15 & 35 & 70 
\end{pmatrix}, $$
so we see that the Pascal matrix contains the Pascal triangle on its antidiagonals.
 
Pascal matrices are ill-conditioned. However, the inverse of the
$n\times n$ Pascal matrix is known explicitly and given in \cite{higham}.
The characteristic polynomial of a Pascal triangle is a reciprocal 
polynomial \cite{higham}.

\begin{thebibliography}{9}
 \bibitem{higham} N.J. Higham, \emph{Accuracy and Stability of Numerical Algorithms},
 2nd ed., SIAM, 2002.
 \end{thebibliography}
%%%%%
%%%%%
\end{document}
