\documentclass[12pt]{article}
\usepackage{pmmeta}
\pmcanonicalname{BernsteinPolynomial}
\pmcreated{2013-03-22 14:15:25}
\pmmodified{2013-03-22 14:15:25}
\pmowner{stitch}{17269}
\pmmodifier{stitch}{17269}
\pmtitle{Bernstein polynomial}
\pmrecord{9}{35705}
\pmprivacy{1}
\pmauthor{stitch}{17269}
\pmtype{Definition}
\pmcomment{trigger rebuild}
\pmclassification{msc}{65D17}
\pmsynonym{Bernstein basis polynomials}{BernsteinPolynomial}
\pmsynonym{Bernstein basis functions}{BernsteinPolynomial}

\endmetadata

% this is the default PlanetMath preamble.  as your knowledge
% of TeX increases, you will probably want to edit this, but
% it should be fine as is for beginners.

% almost certainly you want these
\usepackage{amssymb}
\usepackage{amsmath}
\usepackage{amsfonts}
\usepackage{graphicx}

% used for TeXing text within eps files
%\usepackage{psfrag}
% need this for including graphics (\includegraphics)
% for neatly defining theorems and propositions
%\usepackage{amsthm}
% making logically defined graphics
%%%\usepackage{xypic}

% there are many more packages, add them here as you need them

% define commands here
\begin{document}
The \emph{Bernstein polynomials} of \PMlinkescapetext{degree} $n$ are defined by
$$B_{i}^{n}(t)={n\choose i}t^i (1-t)^{n-i} \quad\quad i=0,1,2,\dots,n$$
where ${n\choose i}$ is the binomial coefficient.

\begin{center}
\includegraphics[scale=.5]{bp}
\end{center}

Bernstein polynomials are used extensively in interpolation theory and in computer graphics. They can be computed efficiently using the de Casteljau's algorithm.

\begin{thebibliography}{9}
\bibitem{faring} Gerald Farin, \emph{Curves and Surfaces for CAGD, A Practical Guide, 5th edition},
Academic Press, 2002.
\end{thebibliography}
%%%%%
%%%%%
\end{document}
