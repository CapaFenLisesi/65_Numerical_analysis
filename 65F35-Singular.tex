\documentclass[12pt]{article}
\usepackage{pmmeta}
\pmcanonicalname{Singular}
\pmcreated{2013-03-22 11:57:38}
\pmmodified{2013-03-22 11:57:38}
\pmowner{Mathprof}{13753}
\pmmodifier{Mathprof}{13753}
\pmtitle{singular}
\pmrecord{11}{30758}
\pmprivacy{1}
\pmauthor{Mathprof}{13753}
\pmtype{Definition}
\pmcomment{trigger rebuild}
\pmclassification{msc}{65F35}
\pmclassification{msc}{15A12}
\pmsynonym{non-invertible}{Singular}
\pmsynonym{singular transformation}{Singular}

\usepackage{amssymb}
\usepackage{amsmath}
\usepackage{amsfonts}
\usepackage{graphicx}
%%%\usepackage{xypic}
\begin{document}
\section{Singular}


An $ m \times n$ matrix $ A$ with entries from a field is called \emph{singular} if its rows or columns are linearly dependent. This is equivalent to the following conditions:
\begin{enumerate}
\item
The nullity of $ A$ is greater than zero ( $ \operatorname{null}(A) > 0$).
\item
The homogeneous linear system $ A\mathbf{x} = 0 $ has a non-trivial solution.
\end{enumerate}

If $m$ = $n$ this is equivalent to the following conditions:
\begin{enumerate}
\item
The determinant $ \det(A)=0$.
\item
The rank of $ A$ is less than $ n$.
\end{enumerate}

%%%%%
%%%%%
%%%%%
\end{document}
