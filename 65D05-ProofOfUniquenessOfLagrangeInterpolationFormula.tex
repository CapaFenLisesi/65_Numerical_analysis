\documentclass[12pt]{article}
\usepackage{pmmeta}
\pmcanonicalname{ProofOfUniquenessOfLagrangeInterpolationFormula}
\pmcreated{2013-03-22 14:09:25}
\pmmodified{2013-03-22 14:09:25}
\pmowner{rspuzio}{6075}
\pmmodifier{rspuzio}{6075}
\pmtitle{proof of uniqueness of Lagrange Interpolation formula}
\pmrecord{10}{35577}
\pmprivacy{1}
\pmauthor{rspuzio}{6075}
\pmtype{Proof}
\pmcomment{trigger rebuild}
\pmclassification{msc}{65D05}
\pmclassification{msc}{41A05}

\endmetadata

% this is the default PlanetMath preamble.  as your knowledge
% of TeX increases, you will probably want to edit this, but
% it should be fine as is for beginners.

% almost certainly you want these
\usepackage{amssymb}
\usepackage{amsmath}
\usepackage{amsfonts}

% used for TeXing text within eps files
%\usepackage{psfrag}
% need this for including graphics (\includegraphics)
%\usepackage{graphicx}
% for neatly defining theorems and propositions
%\usepackage{amsthm}
% making logically defined graphics
%%%\usepackage{xypic}

% there are many more packages, add them here as you need them

% define commands here
\begin{document}
Existence is clear from the construction, the uniqueness is proved by assuming there are two different polynomials $p(x)$ and $q(x)$ that interpolate the points. Then $r(x)=p(x)-q(x)$ has $n$ zeros, $x_1,\ldots, x_n$ and there is a point $x_e$ such that $r(x_e)\neq 0$. $r(x)$ is non-constant with degree $\deg(r(x))\leq n-1$ and has more than $n-1$ solutions, which is a contradiction. Thus there can only be one polynomial.
%%%%%
%%%%%
\end{document}
