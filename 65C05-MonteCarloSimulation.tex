\documentclass[12pt]{article}
\usepackage{pmmeta}
\pmcanonicalname{MonteCarloSimulation}
\pmcreated{2013-03-22 17:20:09}
\pmmodified{2013-03-22 17:20:09}
\pmowner{stevecheng}{10074}
\pmmodifier{stevecheng}{10074}
\pmtitle{Monte Carlo simulation}
\pmrecord{5}{39690}
\pmprivacy{1}
\pmauthor{stevecheng}{10074}
\pmtype{Topic}
\pmcomment{trigger rebuild}
\pmclassification{msc}{65C05}
\pmclassification{msc}{62-00}
\pmclassification{msc}{60-00}
\pmsynonym{Monte Carlo}{MonteCarloSimulation}
\pmsynonym{statistical sampling}{MonteCarloSimulation}
\pmsynonym{stochastic simulation}{MonteCarloSimulation}
\pmrelated{AcceptanceRejectionMethod}

% The standard font packages
\usepackage{amssymb}
\usepackage{amsmath}
\usepackage{amsfonts}

% For neatly defining theorems and definitions
%\usepackage{amsthm}

% Including EPS/PDF graphics (\includegraphics)
%\usepackage{graphicx}

% Making matrix-based graphics
%%%\usepackage{xypic}

% Enumeration lists with different styles
%\usepackage{enumerate}

% Set up the theorem environments
%\newtheorem{thm}{Theorem}
%\newtheorem*{thm*}{Theorem}

\newcommand{\defnterm}[1]{\emph{#1}}

% The standard number systems
\newcommand{\complex}{\mathbb{C}}
\newcommand{\real}{\mathbb{R}}
\newcommand{\rat}{\mathbb{Q}}
\newcommand{\nat}{\mathbb{N}}
\newcommand{\intset}{\mathbb{Z}}

% Absolute values and norms
% Normal, wide, and big versions of the delimeters
\newcommand{\abs}[1]{\lvert#1\rvert}
\newcommand{\absW}[1]{\left\lvert#1\right\rvert}
\newcommand{\absB}[1]{\Bigl\lvert#1\Bigr\rvert}
\newcommand{\norm}[1]{\lVert#1\rVert}
\newcommand{\normW}[1]{\left\lVert#1\right\rVert}
\newcommand{\normB}[1]{\Bigl\lVert#1\Bigr\rVert}

% Inverse functions
\newcommand{\inv}[1]{{#1}^{-1}}

% Differentiation operators
\newcommand{\od}[2]{\frac{d #1}{d #2}}
\newcommand{\pd}[2]{\frac{\partial #1}{\partial #2}}
\newcommand{\pdd}[2]{\frac{\partial^2 #1}{\partial #2}}
\newcommand{\ipd}[2]{\partial #1 / \partial #2}

% Differentials on integrals
\newcommand{\dx}{\, dx}
\newcommand{\dt}{\, dt}
\newcommand{\dmu}{\, d\mu}

% Inner products
\newcommand{\ip}[2]{\langle {#1}, {#2} \rangle}

% Complex numbers
\DeclareMathOperator{\zRe}{Re}
\DeclareMathOperator{\zIm}{Im}
\newcommand{\conjug}[1]{\overline{#1}}

% Calligraphic letters
\newcommand{\sF}{\mathcal{F}}
\newcommand{\sD}{\mathcal{D}}

% Standard spaces
\newcommand{\Hilb}{\mathcal{H}}
\newcommand{\Le}{\mathbf{L}}

% Operators and functions occassionally used in my articles
\DeclareMathOperator{\D}{D}
\DeclareMathOperator{\linspan}{span}
\DeclareMathOperator{\rank}{rank}
\DeclareMathOperator{\lindim}{dim}
\DeclareMathOperator{\sinc}{sinc}

% Probability stuff
\newcommand{\PP}{\mathbb{P}}
\newcommand{\E}{\mathbb{E}}

\begin{document}
\emph{Monte Carlo simulation}
is the use of randomized numerical experiments
to evaluate mathematical expressions.

In the typical problems addressed by Monte Carlo simulation,
the search space or sample space is countably or uncountably infinite.
In contrast to determistic algorithms that sweep a finite subset of points
in the search space in order to derive a solution
to the problem, Monte Carlo simulation randomizes the selection
of points in the hope that good representatives for the solution
of the problem are chosen.

\subsection*{Monte Carlo integration}

The term Monte Carlo simulation most often
refers to integration by randomized methods.
The problem is to evaluate the expectation or integral
\[
\E X = \int X \, d\PP
\]
of a real-valued 
random variable $X$ on a probability space.
The expectation is approximated by taking the sample mean
of independent observations $\{ X_i \}$ with the same distribution
as $X$:
\[
\E X \approx \frac{1}{n} \sum_{i=1}^n X_i\,.
\]
By the strong law of large numbers, the sample mean
converges almost surely, as $n \to \infty$, to the true mean $\E X$.

Typically the distribution of $X$ is not known in closed form;
otherwise we would be able to use the standard numerical quadrature
techniques to compute the one-dimensional integral representing $\E X$,
and one-variable quadrature 
in general tend to work better than the Monte Carlo method.

Rather, the random variable $X = f(Y)$ is expressed as some function $f$ 
of another random variable $Y$, and we know how to compute $f$
and randomized samples of $Y$.
Then 
\[
\E X = \E[ f(Y) ] \approx \frac1n \sum_{i=1}^n f(Y_i)
\]
for independent samples $\{ Y_i \}$
for the random variable $Y$.

The realizations $Y_i$ 
may be obtained by generating random (or pseudo-random)
samples according to the known distribution for $Y$.
Or, in some cases, they may be obtained from pre-tabulated
observations, e.g. based on past observations in the real world.

\subsection*{Bounds on error}

If $X_1, X_2, \dotsc$ are identically and independently distributed,
with mean $m$ and variance $v$,
then
\[
\frac{1}{\sqrt{n}} \sum_{i=1}^n \frac{X_i - m}{\sqrt{v}} 
\]
converges in distribution,
as $n \to \infty$, to the standard normal distribution.
Then for any $\delta > 0$, and for large $n$, we have
\[
\Pr \left( -\delta < \frac{1}{\sqrt n} \sum_{i=1}^n \frac{X_i - m}{\sqrt v}
< \delta \right) \approx 2 \Phi(\delta) - 1\,,
\]
approximating the true distribution with the Gaussian cumulative
distribution function $\Phi$.
Setting $(1-\alpha) \times 100\% = 2\Phi(\delta)-1$ as the required
 confidence level,
we have $\delta = \inv\Phi(1-\alpha/2)$.

Thus, given an observation of the sample mean 
from a run of the Monte Carlo simulation,
with $(1-\alpha) \times 100\%$ confidence,
the true mean $\E X$ is approximately
within
\[
\pm \delta \sqrt{\frac{v}{n}}
=
\pm
\sqrt{\frac{v}{n}} \,
\inv\Phi(1-\alpha/2) 
\]
of the sample mean.

Since the computed sample mean is random,
it is in general not possible to obtain actual error \emph{bounds},
in the usual sense of the word,
as with determistic algorithms.
Instead, the confidence interval size,
or the standard deviation of $(1/n) \sum_{i=1}^n X_i$,
is used as a measure of the error of Monte Carlo simulation;
this is sometimes called a \emph{probabilistic error bound}.

Observe that the probabilistic error bound of 
the Monte Carlo simulation decreases as $1/\sqrt{n}$.

\subsection*{Comparison with other methods}

\begin{description}
\item[Simplicity.]
The chief advantage of Monte Carlo simulation,
compared to the other numerical methods that can solve the same
problem, is that it is conceptually very simple.
It does not require specific knowledge of the form of the solution
or its analytic properties.
Monte Carlo is also relatively easy to implement on a computer.

\item[Slowness.]
The main disadvantage of Monte Carlo integration
is that it is slow.
Many samples may be 
required --- on the order of thousands or even millions ---
to obtain acceptable precision in the answer.
In particular, since the probabilistic error bound
decreases as the reciprocal square root of the number of iterations,
to achieve one more decimal digit of precision
in the answer requires $10^2 = 100$ times more the
number of iterations.
\end{description}

\textbf{Other advantages of Monte Carlo}.
\begin{description}
\item[Independence of dimension.]
The amount of work to obtain the same amount of precision
is independent of the dimension $d$ 
of the underlying random variables.

Thus Monte Carlo integration is practically the only method
to numerically compute high-dimensional integrals.  Traditional
quadrature techniques generally require an amount of work
exponential in the number of dimensions $d$,
since they require sampling a grid in $d$-dimensional space.

On the other hand, Monte Carlo integration is generally
not competitive with quadrature for low-dimensional integration
(e.g. $d=1$ or $d=2$).

\item[Unrestricted choice of functions.]
The functions to integrate with Monte Carlo 
can be practically arbitrary.
No smoothness conditions or boundedness conditions are needed,
for example, providing the integral is finite.

(However, irregularities in the integrand
may impact the accuracy of the result.)

\item[Easily parallelizable.]
Many computer processors can be participating 
in a Monte Carlo simulation simultaneously.  
Each simulation is independent of another.
\end{description}

\textbf{Other disadvantages of Monte Carlo}.
\begin{description}
\item[Error may depend on distribution.]
The estimate of the expectation $\E X$ may be impacted 
severely if the distribution of $X$ is heavily skewed
or has heavier-than-normal tails.
A non-random numerical method
may avoid these deficiencies,
or at least not be as severely impacted.

\item[Difficulty in estimating error.]
There are no hard bounds on the error of the computed result.
The probabilistic error bound, which is essentially based on the
variance, may not be a good measure
of the error for skewed distributions.

\item[Black-box approach.]
With some types of analytical approximation,
one can study the behavior of the solution
if the initial parameters are changed.
This generally is hard to do with 
the black-box approach of Monte Carlo.
\end{description}

\subsection*{Variance reduction}

There are some variance reduction techniques
that can be used to reduce the error in the result
of a Monte Carlo simulation.
However, they generally cannot overcome the 
slow $1/\sqrt{n}$ decrease of the error inherent in Monte Carlo.


\begin{thebibliography}{6}
\bibitem{Gentle}
James E. Gentle. \emph{Random Number Generation and Monte Carlo Methods},
second edition.  Springer, 2003. 
\bibitem{Wiki}
``\PMlinkexternal{Monte Carlo method}{http://en.wikipedia.org/wiki/Monte\_Carlo\_method}''. Wikipedia, The Free Encyclopedia. 
Accessed 27 June 2007.
\end{thebibliography}

%%%%%
%%%%%
\end{document}
